\chapter*{Einleitung}
\label{sec:einleitung}

\section*{Motivation}
\label{sec:motivation}
Das MP3-Format ist ein g�ngiges Dateiformat, um Audiodaten komprimiert
zu speichern. Die Beliebtheit dieses Formats, um Musik �ber das
Internet zu streamen, l�sst sich leicht an Hand der Hunderten von
aktiven MP3-Musikradios feststellen (siehe
z.B. \verb|http://www.shoutcast.com/|).

Diese Musikradios benutzen HTTP (\emph{Hypertext Transport Protocol})
als Transportprotokoll. Dieses Protokoll hat sich aufgrund der
einfachen Implementierung von Client und Server durchgesetzt, da die
Synchronisation von Sender und Empf�nger durch das unterliegende TCP
�bernommen wird. Allerdings hat HTTP mehrere Nachteile:
\begin{itemize}
\item Erh�hte Netzwerkbelastung durch TCP,
\item Keine M�glichkeit, Daten gleichzeitig an mehrere Empf�nger zu
  senden (Multicast),
\item L�ngere Aussetzer bei Netzwerklatenzen oder Paketverlusten.
\end{itemize}

Es gibt mehrere Ans�tze um MP3-Daten in Multicastnetzwerken zu
versenden.  In dieser Studienarbeit werden zwei Verfahren untersucht,
mit denen MP3-Daten �ber RTP (\emph{Real Time Protocol},
\cite{RFC1889}) versendet werden. Das Erste dieser Verfahren, das in
RFC-2250 \cite{RFC2250} definiert wird, schreibt vor, MP3-Frames als
Nutzdaten eines RTP Paketes zu benutzen.  Dieses Verfahren hat
allerdings wegen einer Feinheit im MP3-Format einen Nachteil: Bei
einzelnen Paketverlusten sind beim Empf�nger gleich mehrere Frames
unlesbar.  Um dieses Problem zu korrigieren wurde das in RFC-3119
\cite{RFC3119} beschriebene Verfahren vorgeschlagen: MP3-Frames werden
erst in unabh�ngige Dateneinheiten konvertiert, die dann als Nutzdaten
eines RTP-Paketes gesendet werden.

Keines dieser Verfahren bietet jedoch eine M�glichkeit, Paketverluste
zu beheben (bei dem Verfahren von RFC-3119 werden die Folgen eines
Paketverlustes nur minimiert).

\section*{Aufgabenstellung}
\label{sec:aufgabenstellung}

Im Rahmen dieser Studienarbeit wurden die oben erw�hnten
Streamingverfahren (also HTTP-Streaming, RFC-2250-Streaming und
RFC-3119-Streaming) implementiert.  Anschliessend wurde ein
Fehlerkorrekturverfahren (basierend auf das Verfahren von Luigi Rizzo
\cite{FEC}) f�r MP3-Streaming entworfen und implementiert.

Im \ref{sec:uebersicht}. Kapitel dieser Studienarbeit werden die
Struktur einer MP3-Datei und das MP3-Kodierungs{-}verfahren kurz
vorgestellt, sowie die allgemeine Problemstellung des �bertragen von
MP3-Daten erl�utert. Im \ref{sec:streaming}. Kapitel werden dann die
einzelnen MP3-Streaming-Verfahren im Detail vorgestellt,
einschlie�lich einer Erweiterung der RFC-2250 und RFC-3119 Verfahren
zum Schutz des MP3-Streams vor F�lschungen. Im \ref{sec:transcoding}.
Abschnitt sind die Algorithmen, die beim Dekodieren und Kodieren von
MP3-Str�men zum Einsatz kommen, dokumentiert. Im \ref{sec:fec}.
Kapitel wird das Verfahren, das zum Kodieren und Dekodieren von
redundanter Fehlerkorrekturinformation eingesetzt wird beschrieben.
Weiterhin wird das implementierte Protokoll zum Versenden von FEC
kodierten MP3-Daten dokumentiert.  Anschliessend wird im
\ref{sec:implementation}. Kapitel kurz auf die Details der
Implementierung eingegangen. Im \ref{sec:zusammenfassung}. Kapitel
werden dann die Ergebnisse der Studienarbeit zusammengefasst und m�gliche
Erweiterungen vorgestellt. Im Appendix \ref{sec:code} ist der
vollst�ndige, dokumentierte Code der implementierten Programme zu
finden.

