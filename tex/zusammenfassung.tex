\chapter{Zusammenfassung}
\label{sec:zusammenfassung}

\section{Schlussfolgerungen}
\label{sec:schlussfolgerung}
In dieser Studienarbeit wurden die existierenden Verfahren zum
Streamen von MP3-Daten vorgestellt, sowie auf ihre Vorteile und
Nachteile eingegangen.

Alle Verfahren wurde nachimplementiert, so dass sie auch in der Praxis
getestet werden konnten. Zwei dieser Verfahren, die MP3-Daten �ber RTP
versenden, wurden erweitert um ein sicheres Versenden von MP3-Daten zu
er\-m�\-gli\-chen (es ist einem Angreifer nicht mehr m�glich, fremde
MP3-Daten in einen MP3-Strom einzuf�gen). Die entwickelten
Softwaremodule konnten auch anderweitig eingesetzt werden, um ein
Werkzeug zum sauberen Schneiden von MP3-Dateien zu realisieren.

Weiter wurde ein Protokoll entworfen, mit dem MP3-Daten redundant
kodiert und versendet werden k�nnen, um Aussetzer bei Paketverlusten
im unterliegenden Kommunikationsmedium (hier UDP) zu vermeiden. Dieses
Protokoll wurde auch implementiert, indem das paketbasierte
FEC-Verfahren von Luigi Rizzo verwendet wurde.

\section{M�gliche Erweiterungen}
\label{sec:erweiterungen}
Das urspr�ngliche Ziel dieser Studienarbeit war die Implementierung
einer autoadaptiven Fehlerkorrektur f�r MP3-Streaming. Bei erh�hter
Redundanzkodierung sollten die MP3-Daten niederskaliert werden, um die
Netzwerkbelastung in Grenzen zu halten. Allerdings war die
Implementierung eines MP3-Transkodierers zu aufwendig, da die
Dokumentation zu dem MP3-Format sehr sp�rlich ist, und das Debuggen
solcher Funktionalit�ten sehr schwer zu gestalten ist. Weiterhin war
das vorgeschlagene Verfahren, die Huffmankodierten Daten des MP3
Stromes runterzukodieren zu ungenau, um eine bestimmte Zielbitrate zu
erreichen. Ein praktischerer Ansatz w�re gewesen, einen existierenden
MP3-Encoder zu benutzen, der als Eingangsdaten einen MP3-Strom
benutzen kann. Mittlerweile kann die Lame Softwarebibliothek
MP3-Str�me umskalieren, so dass im Grunde genommen der Implementierung
einer autoadaptiven Fehlerkorrektur mit Transkodierung des MP3-Stromes
nichts im Wege steht.

Weiterhin ist das offene Vorbis Format sehr einfach handzuhaben, und
unterst�tzt auch ``Bitpeeling'', d.h. die M�glichkeit, Teile eines
Frames wegzulassen ohne die Dekodierbarkeit der Audiodaten zu
gef�hrden (es wird nur ein Qualit�tsverlust in Kauf genommen). Mit
diesem Verfahren lassen sich die Audiodaten ohne intensive Rechnungen
an die Netzwerkbandbreite anpassen. Das Vorbis Format benutzt auch
unabh�ngige ADUs als Frames, so dass die Konvertierung von Frames nach
ADUs wegf�llt. Mittlerweile wurde ein HTTP-Vorbis Streamer
implementiert.

Die Paketverlustrate �ndert sich meistens rasch, so dass eine
statische Einstellung der Fehlerkorrekturparameter nicht vertretbar
ist. Dazu m�sste im Fall von Multicast Streaming ein Feedbackprotokoll
benutzt werden, um die Fehlerkorrektur den W�nschen der meisten
Clients anzupassen. Dazu k�nnte man einer der vielen vorgeschlagenen
Feedbackprotokolle benutzen. Weiterhin sollten die Netzwerkprogramme
auf IPv6 portiert werden.

Letztendlich gibt es bei der implementierten Software noch viele
L�cken, wie z.B. eine gute Ausnahmebehandlung. In den meisten F�llen
wird im Moment bei schwerwiegenden Fehlern einfach der Prozess
quittiert.
